El aprendizaje autom\'atico es una rama de la inteligencia artificial que tiene como objetivo el desarrollo de algoritmos que permitan a la m\'aquina aprender de la experiencia. \\

Al ser el objetivo del trabajo conseguir una predicci\'on del momento transverso de muones \textit{high-$p_{T}$} simulados a partir de las se\~nales que dejan a su paso por el detector CMS, nos centraremos en el aprendizaje autom\'atico supervisado, donde en el proceso de aprendizaje (entrenamiento) del modelo matem\'atico se le proporcionan tanto las caracter\'isticas del mu\'on (datos de entrada) como el momento real asignado en la simulaci\'on de la part\'icula, con el fin de que en esta etapa el modelo pueda encontrar correlaciones y patrones de comportamiento. Por otra parte, en el proceso de testeo el algoritmo ha de ser capaz de poder realizar predicciones tomando datos de entrada de muones con momento transverso desconocido. \\

Dentro de los distintos tipos de algoritmos de aprendizaje autom\'atico destacan las redes neuronales artificiales, cuyo funcionamiento se basa en la redes neuronales biol\'ogicas que propagan la informaci\'on en el cerebro: \\

profunda o DNN \\


En f\'isica de altas energ\'as, el uso de las redes neuronales profundas se ha extendido en los \'ultimos a\~nos...
