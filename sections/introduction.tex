
El objeto principal de estudio de este trabajo es el mu\'on, una part\'icula elemental cargada con spin 1/2 y con masa aproximadamente 200 veces mayor que el electr\'on, siendo adem\'as una part\'icula inestable con un tiempo de vida de 2.2 $\mu s$, que es elevado en comparaci\'on con otras part\'iculas que poseen esta propiedad. \\

En este trabajo nos centraremos espec\'ificamente en la detecci\'on y reconstrucci\'on de muones en el detector CMS (del ingl\'es \textit{Compact Muon Solenoid}), situado en el gran colisionador de hadrones LHC (del ingl\'es \textit{Large Hadron Collider}) del CERN, en la ciudad suiza de Ginebra. \\
La mayor parte de los muones medidos por CMS provienen t\'ipicamente de desintegraciones de quarks top, de hadrones, o de desintegraciones lept\'onicas de bosones $Z$ o $W$~\cite{PhysRevD.98.030001}. Estos muones se caracterizan normalmente por tener un momento transverso ($p_{T}$), la componente del momento del mu\'on en el plano perpendicular a la direcci\'on de choque de los protones, inferior a 200 GeV y se categorizan como muones de bajo momento o bajo-$p_{T}$. \\

Por otra parte, los muones de alto momento o alto-$p_{T}$ pueden tener como origen procesos f\'isicos at\'ipicos como la desintegraci\'on de part\'iculas ex\'oticas m\'as all\'a del Modelo est\'andar de la f\'isica de part\'iculas~\cite{gaillard1999standard} como bosones $Z'$ o $W'$~\cite{CMS-PAS-EXO-19-019,2017278}, con masas en la escala del TeV. El descubrimiento de alguna de estas part\'iculas ser\'ia un indicativo directo de nueva f\'isica, por lo que medir las propiedades de estos muones de la manera m\'as precisa posible en el detector es de vital importancia. Concretamente, el objetivo principal del trabajo consiste en aplicar una t\'ecnica de regresi\'on basada en una red neuronal profunda o DNN (del ingl\'es \textit{Deep Neural Network})~\cite{Schmidhuber:2015} para estimar el momento transverso de los muones de alto-$p_{T}$ con precisi\'on. \\

Experimentalmente, la medida del momento de los muones de alto-$p_{T}$ plantea varias dificultades. \\
Primero, hay que tener en cuenta que la resoluci\'on de la medida del $p_{T}$ a partir de la traza empeora cuando el momento del mu\'on aumenta, debido a que el $p_{T}$ se mide a trav\'es de la relaci\'on entre curvatura y momento que establece la fuerza de Lorentz.\\
De esta manera, en presencia de un campo magn\'etico uniforme $B$, y con un radio de curvatura de la traza $r$, el momento transverso $p_{T}$ de un mu\'on con carga $q$ se puede expresar como:

\begin{equation}
  p_{T}[\text{GeV}] = 0.3 \times B[\text{T}] \times r[\text{m}] \times q
\label{eq:pTvsRadius}
\end{equation}

El campo magn\'etico dentro del solenoide de CMS es pr\'acticamente uniforme y conocido con gran precisi\'on ($B$ = 3.8 T), mientras que el radio de curvatura se relaciona con la longitud del arco $L$ y la distancia sagitta $s$ definida en la Figura~\ref{fig:SagittaDef} a trav\'es de:

\begin{equation}
  R[\text{m}]\approx L[\text{m}]^{2}/8s[\text{m}]
\label{eq:RadiusvsSagitta}
\end{equation}

Siendo esta aproximaci\'on v\'alida para $L/r \ll 1$. \\

Combinando ahora las ecuaciones \eqref{eq:pTvsRadius} y \eqref{eq:RadiusvsSagitta}, se obtiene:

\begin{equation}
  s[\text{m}]\approx (0.3 B [\text{T}] L[\text{m}]^{2}/8) (q/p_{T}[\text{GeV}]) =  (0.3 BL^{2}/8) \times (q/p_{T}),
\label{eq:SagittavsPt}
\end{equation}

Se observa que $s$ es inversamente proporcional al momento transverso, por lo que para mejorar la resoluci\'on en la medida del $p_{T}$ en los casos con sagittas peque\~nas, las trazas de los muones en CMS se reconstruyen en distintos subdetectores separados a varios metros del punto de colisi\'on (como se detallar\'a en la Secci\'on~\ref{sec:CMS}) para as\'i tener trazas de mayor longitud y por consiguiente mayores valores de $s$. \\
Hay que tener en cuenta que medir $s$ con precisi\'on puede ser un gran desaf\'io; por ejemplo, un mu\'on con $p_{T}$ = 200 GeV que atraviesa CMS presenta un valor de sagitta del orden del mil\'imetro, lo que requiere contar con detectores de muones con una resoluci\'on espacial excelente. Para muones en la escala del TeV, incluso las incertidumbres sistem\'aticas en el posicionamiento de los detectores pueden introducir un sesgo en la medida final. \\

\begin{figure}[h]
\centering
\includegraphics[width=0.30\textwidth]{figures/curvaturesketch.png}
\caption{Definici\'on de la distancia saggita, $s$, a partir de la longitud de la traza reconstruida $L$ y de su radio $r$.}
\label{fig:SagittaDef}
\end{figure}

En segundo lugar, cuando los muones de alto momento atraviesan las componentes m\'as densas del detector, formadas por hierro principalmente, no s\'olo sufren p\'erdida de energ\'ia por ionizaci\'on, sino que tambi\'en sufren otros procesos como la producci\'on de pares, Bremsstrahlung, o interacciones fotonucleares, que alteran su energ\'ia haciendo m\'as dif\'icil el proceso de reconstrucci\'on. \\
En la Figura~\ref{fig:dEdX} se muestra la dependencia de la p\'erdida de energ\'ia del mu\'on por unidad de distancia $dE/dx$ al atravesar distintos medios como funci\'on de su energ\'ia. 

\begin{figure}[h]
\centering
\includegraphics[width=0.60\textwidth]{figures/dEdx.png}
\caption{Media de la p\'erdida de energ\'ia por ionizaci\'on y por radiaci\'on del mu\'on en hidr\'ogeno, hierro y uranio, como funci\'on de su energ\'ia. En el caso del hierro, se separan las contribuciones para la producci\'on de pares, Bremsstrahlung e interacciones fotonucleares. Figura tomada de \cite{Tanabashi:2018oca}.}
\label{fig:dEdX}        
\end{figure}

Se observa que la energ\'ia cr\'itica para el hierro, $E^{iron}_{c}$, donde la energ\'ia de ionizaci\'on (en marr\'on) es igual a la suma de todas las p\'erdidas radiativas (en morado), ocurre aproximadamente a 300 GeV. Como consecuencia, la principal fuente de p\'erdida de energ\'ia para un mu\'on con $E>E^{iron}_{c}$ que viaja por hierro a trav\'es de los distintos subdetectores de CMS es debida a radiaci\'on electromagn\'etica fruto de la producci\'on de electrones y fotones. Esta radiaci\'on electromagn\'etica se manifiesta en el detector como una cascada de part\'iculas que producen se\~nales adicionales en los detectores, y puede incluso cambiar la direcci\'on de la trayectoria del mu\'on, afectando negativamente a la reconstrucci\'on de su traza y degradando por consiguiente la medida de su momento. \\

Los algoritmos de reconstrucci\'on de trazas que son utilizados para la asignaci\'on del momento transverso a los muones de alto-$p_{T}$ en CMS evitan lidiar con casos de emisi\'on de cascadas, como se explicar\'a con m\'as detalle en la Secci\'on~\ref{sec:current_assignment}. De esta manera, por ejemplo, si se encuentran varios impactos en un subdetector que puedan ser indicativo de que una cascada ha tenido lugar, t\'ipicamente se ignoran estas se\~nales a la hora de reconstruir la traza del mu\'on. \\

El objetivo de este trabajo, por contrapartida, es recopilar toda la informaci\'on posible de la trayectoria de los muones a su paso por CMS (incluyendo las se\~nales provenientes de cascadas electromagn\'eticas), utilizando para ello muones simulados altamente energ\'eticos con $p_{T}$ generado conocido (el momento real asignado en la simulaci\'on de la partícula), y entrenar posteriormente una DNN que haga regresi\'on al $p_{T}$ reconstruido, de manera que se consiga una asignaci\'on de momento transverso mejor que el proporcionado por los algoritmos centrales de CMS.\\

Para cuantificar la calidad de la asignaci\'on del momento transverso, se har\'a uso de la variable $R$, definida como:

\begin{equation}
  R = \dfrac{\lvert p_{T}^{GEN} - p_{T}^{RECO}\rvert}{p_{T}^{GEN}}
\label{eq:R}
\end{equation}

Donde $p_{T}^{GEN}$ hace referencia al momento generado del mu\'on y $p_{T}^{RECO}$ al momento reconstruido a trav\'es de su traza en el detector. \\
Cabe destacar que el valor central de la distribuci\'on de $R$ da cuenta del sesgo existente en la estimaci\'on del momento, mientras que su anchura da cuenta de la resoluci\'on del sistema.\\

La presente memoria est\'a estructurada en las siguientes secciones: \\
En la Secci\'on~\ref{sec:CMS} se describir\'a brevemente el dispositivo experimental utilizado: el detector CMS. La Secci\'on~\ref{sec:ML} contiene una breve introducci\'on al aprendizaje autom\'atico y a las redes neuronales profundas, as\'i como sus aplicaciones m\'as habituales en f\'isica de part\'iculas. En la Secci\'on~\ref{sec:current_assignment} se presentar\'an los algoritmos actuales utilizados para la asignaci\'on de momento transverso. \\ Posteriormente, en la Secci\'on~\ref{sec:methodology}, se describir\'a el m\'etodo propuesto para la selecci\'on de muones y segmentos sobre la muestra de simulaci\'on utilizada. \\
Finalmente, en la Secci\'on~\ref{sec:training} se detallar\'a la arquitectura de la red utilizada y las variables de entrenamiento, y en las Secciones~\ref{sec:results} y \ref{sec:conclusions} se mostrar\'an los resultados finales y las conclusiones del estudio.
