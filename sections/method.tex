
Dado que el objetivo del trabajo es mejorar la asignaci\'on de momento para muones high-$p_{T}$, especialmente para los casos con showering, el primer aspecto que se ha de tener en cuenta es recopilar toda la informaci\'on posible del mu\'on (incluyendo los hits que provienen de las cascadas), en contraposici\'on a la estrategia de los distintos refits descritos en la Secci\'on~\ref{sec:current_assignment}, que tratan de encontrar aquellas c\'amaras donde es probable que haya tenido lugar una cascada para despu\'es eliminar la informaci\'on de dichas c\'amaras a la hora de hacer la reconstrucci\'on de la traza del mu\'on.

De esta manera, el procedimiento que se va a seguir para recoger toda la informaci\'on posible es extrapolar la traza del mu\'on medida en el tracker a las c\'amaras de muones, para despu\'es recoger todo el conjunto de hits que se encuentren en torno al punto extrapolaci\'on recogiendo as\'i todas las señales que produzca la posible cascada. \\

En la presente secci\'on se detallar\'an las herramientas y metodolog\'ia utilizadas para el tratamiento de los datos.


\subsection{Herramientas utilizadas para el an\'alisis}\label{sec:tools}

Las herramientas utilizadas en el proceso de an\'alisis de los datos son las siguientes:

\begin{itemize}

\item CMSSW~\cite{cmssw} es una colecci\'on de software abierto utilizado principalmente para la hacer simulaci\'on, calibraci\'on, alineamiento, y reconstrucci\'on de objetos f\'isicos para el posterior an\'alisis de los datos. Para este trabajo se ha creado un c\'odigo en C++ que puede encontrarse en el repositorio~\cite{analyzer}. Este c\'odigo usa los m\'odulos de CMSSW para hacer la selecci\'on de muones y segmentos que se describir\'a en la secci\'on~\ref{sec:selection}. 

\item ROOT~\cite{root} es el marco de trabajo m\'as utilizado f\'isica de altas energ\'ias para el procesado de datos, an\'alisis estad\'istico, y herramientas de visualizaci\'on y almacenamiento de los datos. ROOT est\'a basado en programaci\'on orientada a objetos y escrito en C++, y se usar\'a como formato de los datos de entrada y salida del c\'odigo de selecci\'on~\cite{analyzer}.

\item Pandas~\cite{mckinney-proc-scipy-2010} (del ingl\'es \textit{Python Data Analysis Library}), es una librer\'ia de Python que ofrece gran rendimiento en el procesado de estructuras tabulares de datos. En este caso, se usar\'a Pandas, previa transformaci\'on de los datos en formato ROOT a csv, para el procesado de los segmentos seleccionados (operaciones de limpiado de segmentos, agregaciones, construcci\'on de las variables de entrenamiento...etc). Los m\'odulos creados para el procesado pueden encontrarse en \cite{processor}.

\end{itemize}


El proceso l\'ogico de trabajo en cuanto a las herramientas utilizadas y al formato de los datos se muestra en el siguiente diagrama de fases: \\

\begin{center}
\smartdiagramset{border color=none,
   text width=3cm, font=\fontsize{10pt}{12pt}\selectfont,
   module x sep=4.0,     
   back arrow disabled=true}
\smartdiagram[flow diagram:horizontal]{Datos de simulaci\'on de entrada (formato ROOT),
  Selecci\'on de muones y segmentos (c\'odigo \cite{analyzer} en C++ con salida en formato ROOT),
  Tranformaci\'on de ROOT a texto (reader.py en \cite{processor}),
  Procesado de datos con Pandas: limpiado y construcci\'on de variables (doStep1.py y step2.py en \cite{processor})}
\end{center}

\subsection{Selecci\'on de muones y segmentos}\label{sec:selection}

En los datos de entrada, Los muones reconstruidos por evento se almacenan en colecciones de ROOT, que funcionan a modo de contenedor de informaci\'on.

Las c\'amaras DT y CSC se componen de varias capas, y las señales o hits que dejan los muones a su paso se reconstruyen en cada cada de ellas. A partir de estos hits por c\'amara, se construyen trazas rectas denominadas segmentos dentro de cada c\'amara DT o CSC. Al igual que en el caso de los muones, los segmentos en un evento se almacenan tambi\'en en colecciones de ROOT.



\subsection{Procesado y construcci\'on de variables}\label{sec:procesado}

FILLME

\subsection{Muestra de simulaci\'on utilizada}\label{sec:sample}

Se han generado 100000 eventos de $Z'$ con $m_{Z'}$ = 5000 GeV fruto de la colisiones prot\'on-prot\'on a una energ\'ia de centro de masas de 13 TeV (condiciones del acelerador LHC) mediante simulaci\'on de Monte-Carlo utilizando el programa MadGraph5~\cite{Alwall:2014hca}. Se impone que las part\'iculas $Z'$ generadas se desintegren a un par de muones $\mu^{+}\mu^{-}$ y se simula el paso de los muones por el detector CMS con el paquete Geant4~\cite{Agostinelli:2002hh}. De esta manera se tiene una muestra con estad\'istica razonable de muones high-$p_{T}$ con $p_{T}$ generado conocido que se usar\'a para el entrenamiento de la DNN. \\

En la Figura~\ref{fig:data_pt} se muestran las distribuciones del $p_{T}$ generado y el $p_{T}$ proporcionado por el algoritmo TuneP de todos los muones de la muestra que pasan la selecci\'on detallada en la Secci\'on~\ref{sec:selection}.

\begin{figure}[h]
\centering
\includegraphics[width=1.0\textwidth]{figures/data_pt.png}
\caption{Distribuciones del momento transverso de los muones de la muestra de simulaci\'on utilizada. Izquierda: $p_{T}$ generado. Derecha: $p_{T}$ proporcionado por el algoritmo TuneP.}
\label{fig:data_pt}        
\end{figure}


\subsection{Distribuciones de control}\label{sec:plots}

FILLME
