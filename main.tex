\documentclass[a4paper,12pt]{article}
\usepackage[a4paper,top=3cm,bottom=3cm,left=3cm,right=3cm,marginparwidth=1.75cm]{geometry}
\usepackage{amsmath}
\usepackage{mathtools}
\usepackage{graphicx}
\usepackage{hyperref}
\usepackage{smartdiagram}
\usepackage{caption}
\usepackage{subfiles}
\usepackage[spanish,es-tabla]{babel}

\setlength{\parindent}{0pt}
\usepackage{xcolor}

\begin{document}


\begin{titlepage}
\centering
\includegraphics[width=0.25\textwidth]{figures/logo_UC.jpg} 
%\hspace{20pt} 
\par 
\vspace{1cm}
{\LARGE \textbf{Facultad de Ciencias} \par}
\vspace{1.5cm}
%English title
{\LARGE\bfseries T\'ecnicas de aprendizaje autom\'atico profundo para la asignaci\'on de momento a muones altamente energ\'eticos en el experimento CMS del LHC}
\vspace{0.6cm}
\\
%Spanish title
{\LARGE (Deep learning techniques for muon momentum assignment with the CMS experiment at LHC) \par}
\vspace{2.6cm}
{\scshape\large Trabajo de fin de M\'aster \\ para acceder al \par}
\vspace{0.3cm}
{\scshape\Large \textbf{M\'ASTER EN \\ CIENCIA DE DATOS} \par}
\begin{flushright}
\vspace{2.6cm}
{\large Autor : Pedro Jos\'e Fern\'andez Manteca \par}
{\large Director : Pablo Mart\'inez Ru\'iz del \'Arbol\\}
{\large Co-director : Alicia Calder\'on Taz\'on\\}
\vspace{0.5cm}
{\large Septiembre - 2020\par}
\vfill
\end{flushright}
\end{titlepage}


\newpage

\null\hfill\begin{tabular}[t]{l@{}}
  \textit{A Alicia y Pablo por el tiempo que} \\
  \textit{hab\'eis invertido en mi aprendizaje.} \\
  \textit{Espero que sigamos trabajando juntos} \\
  \textit{durante muchos a\~nos m\'as.} \\
  \textit{Gracias.} \\
\end{tabular}


%% \section*{Agradecimientos}

%% HOLA

\newpage

\tableofcontents

\newpage

\begin{abstract}
El objetivo de este trabajo es aplicar t\'ecnicas de aprendizaje autom\'atico profundo supervisado para la asignaci\'on de momento transverso a muones altamente energ\'eticos en el experimento CMS (\textit{Compact Muon Solenoid}) del LHC (\textit{Large Hadron Collider}) mediante regresi\'on, teniendo como principal meta mejorar los resultados del procedimiento actual, y de ser as\'i plantear incluir este tipo de metodolog\'ias de manera oficial en la futura toma de datos del experimento (Run 3). \\

Para el entrenamiento se ha utilizado una muestra de muones simulados, de manera que el algoritmo predictivo se alimenta de la informaci\'on que estos muones dejan a su paso por el detector CMS. \\
Los resultados obtenidos muestran una mejora del 26\% en la resoluci\'on del momento transverso para los muones con 1200 $\leq p_{T} \leq$ 2000 GeV y $\lvert \eta \rvert <$ 0.9. \\

\textbf{Palabras clave:} aprendizaje autom\'atico profundo, muones altamente energ\'eticos, CMS, LHC. \\ \\ \\

\end{abstract}

\renewcommand{\abstractname}{Abstract}
\begin{abstract}
The aim of this work is to apply supervised deep machine learning techniques for the transverse momentum assignment to highly energetic muons in the CMS (\textit{Compact Muon Solenoid}) experiment at LHC (\textit{Large Hadron Collider}) through regression, with the main goal of improving the results of the current procedure, and if so, consider including this kind of methodologies for the future data taking of the experiment (Run 3). \\

For the training, a sample of simulated muons has been used, so that the predictive algorithm feeds on the information that these muons leave as they pass through the CMS detector. \\
The results obtained show an improvement of 26\% in the resolution of the transverse momentum for the muons with 1200 $ \leq p_ {T} \leq $ 2000 GeV and $ \lvert \eta \rvert <$ 0.9. \\

\textbf{Key words:} deep learning, highly energetic muons, CMS, LHC. \\

\end{abstract}


\newpage

\section{Introducci\'on}\label{sec:intro}

\subfile{sections/introduction}

\newpage

\section{El experimento CMS}\label{sec:CMS}

\subfile{sections/CMS}

\newpage

\section{Aprendizaje autom\'atico y su uso en f\'isica de altas energ\'ias}\label{sec:ML}

\subfile{sections/ML}

\newpage

\section{Asignaci\'on de momento transverso en el experimento CMS}\label{sec:current_assignment}

\subfile{sections/CurrentAssignment}

\newpage

\section{M\'etodo propuesto}\label{sec:methodology}

\subfile{sections/method}

\newpage

\section{Implementaci\'on y entrenamiento de la DNN}\label{sec:training}

\subfile{sections/training}

\newpage

\section{Resultados}\label{sec:results}

\subfile{sections/results}

\newpage

\section{Conclusiones}\label{sec:conclusions}

\subfile{sections/conclusions}

\newpage

\bibliographystyle{unsrt}
\bibliography{references}

\end{document}
